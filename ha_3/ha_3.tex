\documentclass[12pt]{article}

\usepackage{tikz} % картинки в tikz
\usepackage{microtype} % свешивание пунктуации

\usepackage{array} % для столбцов фиксированной ширины

\usepackage{indentfirst} % отступ в первом параграфе

\usepackage{sectsty} % для центрирования названий частей
\allsectionsfont{\centering}

\usepackage{amsmath} % куча стандартных математических плюшек

\usepackage{comment}
\usepackage{amsfonts}

\usepackage{verbatim}

\usepackage{graphicx} % for png images

\usepackage[colorlinks=true, linkcolor=blue]{hyperref}

\usepackage[top=2cm, left=1cm, right=1cm, bottom=2cm]{geometry} % размер текста на странице

\usepackage{lastpage} % чтобы узнать номер последней страницы

\usepackage{enumitem} % дополнительные плюшки для списков
%  например \begin{enumerate}[resume] позволяет продолжить нумерацию в новом списке
\usepackage{caption}

\usepackage{hyperref} % гиперссылки

\usepackage{multicol} % текст в несколько столбцов


\usepackage{fancyhdr} % весёлые колонтитулы
\pagestyle{fancy}
\lhead{Наука о данных-ВШЭ}
\chead{}
\rhead{Домашнее задание 3}
\lfoot{2020-05-21}
\cfoot{}
\rfoot{}
\renewcommand{\headrulewidth}{0.4pt}
\renewcommand{\footrulewidth}{0.4pt}



\usepackage{todonotes} % для вставки в документ заметок о том, что осталось сделать
% \todo{Здесь надо коэффициенты исправить}
% \missingfigure{Здесь будет Последний день Помпеи}
% \listoftodos --- печатает все поставленные \todo'шки


% более красивые таблицы
\usepackage{booktabs}
% заповеди из докупентации:
% 1. Не используйте вертикальные линни
% 2. Не используйте двойные линии
% 3. Единицы измерения - в шапку таблицы
% 4. Не сокращайте .1 вместо 0.1
% 5. Повторяющееся значение повторяйте, а не говорите "то же"


\usepackage{fontspec}
\usepackage{polyglossia}

\setmainlanguage{russian}
\setotherlanguages{english}

\usepackage{xeCJK} % японские буквы
\setCJKmainfont{Noto Sans CJK JP} % и шрифт к ним :)

% download "Linux Libertine" fonts:
% http://www.linuxlibertine.org/index.php?id=91&L=1
\setmainfont{Linux Libertine O} % or Helvetica, Arial, Cambria
% why do we need \newfontfamily:
% http://tex.stackexchange.com/questions/91507/
\newfontfamily{\cyrillicfonttt}{Linux Libertine O}

\AddEnumerateCounter{\asbuk}{\russian@alph}{щ} % для списков с русскими буквами
\setlist[enumerate, 2]{label=\asbuk*),ref=\asbuk*}

%% эконометрические сокращения
\DeclareMathOperator{\Cov}{Cov}
\DeclareMathOperator{\Corr}{Corr}
\DeclareMathOperator{\Var}{Var}
\DeclareMathOperator{\E}{E}
\def \hb{\hat{\beta}}
\def \hs{\hat{\sigma}}
\def \htheta{\hat{\theta}}
\def \s{\sigma}
\def \hy{\hat{y}}
\def \hY{\hat{Y}}
\def \v1{\vec{1}}
\def \e{\varepsilon}
\def \he{\hat{\e}}
\def \z{z}
\def \hVar{\widehat{\Var}}
\def \hCorr{\widehat{\Corr}}
\def \hCov{\widehat{\Cov}}
\def \cN{\mathcal{N}}
\def \P{\mathbb{P}}
\newcommand \id {\mathrm{id}\_\mathrm{for}\_\mathrm{online}}

\begin{document}



\vspace{20mm}

\textbf{Заповеди:}

\vspace{5mm}

\begin{enumerate}  
\item Be good, drink milk and \href{https://www.youtube.com/watch?v=jyxSFfBKMxQ}{think of Russia} (c)
\item Обязательно фиксируйте зерно генератора случайных чисел в экспериментах. 
При перезапуске кода значения не должны меняться :)
\item Закомитьте задание в виде \verb|.ipynb| файла на гитхаб \url{https://classroom.github.com/a/3KL2oSuX}.
\item Вверху файла подпишите фамилию, имя и группу.
\item Дедлайн: 7 июня 21:00 без штрафа, 8 июня 21:00 со штрафом 50\%.
\item Ещё потребуется загрузить задачки в контест для проверки на плагиат ;)
\end{enumerate}


\section*{Задачи}
\begin{enumerate}
\item Парадокс инспектора. 

Автобусы отходят от автостанции с 8:00 до 20:00. Первый автобус отходит ровно в 8:00. 
Затем интервалы между автобусами случайны, независимы и равновероятно равны либо 5-и, либо 10-и
минутам. 
Будем считать, что за минуту на автостанцию приходит ровно один пассажир, 
и все пассажиры едут ближайшим автобусом.

Проведите $10^4$ экспериментов и с их помощью:

\begin{enumerate}
    \item {[10]} Постройте гистограмму количества автобусов, отошедших от автостанции за сутки. 
    Похоже ли визуально распределение на нормальное?
    \item {[10]} Инспектор Тимон выбирает равновероятно один из всех автобусов отошедших от автостанции за сутки. 
    Постройте гистограмму числа пассажиров на этом автобусе. Оцените математическое ожидание и дисперсию этого числа.
    \item {[10]} Инспектор Пумба приходит на автостанцию в случайный момент времени, равномерный от 8:00 до 20:00 и 
    садится в первый пришедший автобус. Постройте гистограмму числа пассажиров на этом автобусе. Оцените математическое ожидание и дисперсию этого числа.
    \item {[10]} Как изменятся ответы на эти вопросы, если время между автобусами будет экспоненциально со средним в 10 минут?
\end{enumerate}

\newpage
\item Парадокс Хуана Мануэля Родригеса Паррондо.

У Атоса, Портоса и Арамиса по 1000 франков. 

Атос постоянно ходит в казино А, где каждый раз выигрывает один франк с вероятностью $0.49$ и проигрывает 
один франк с вероятностью $0.51$.

Портос ходит в казино Б, где ситуация интереснее :) 
Если богатство посетителя кратно трём, то посетитель выигрывает франк с вероятностью $0.09$ 
и проигрывает один франк с вероятностью $0.91$. 
Если богатство посетителя не кратно трём, то посетитель выигрывает франк с вероятностью $0.74$ 
и проигрывает один франк с вероятностью $0.26$. 

Арамис каждый раз выбирает казино А или казино Б равновероятно. 

Проведите $10^4$ симуляций эволюции благосостояния Атоса, Портоса и Арамиса.
Каждая симуляция предусматривает $1000$ посещений казино. 
Все три игрока все деньги носят с собой и ни на что не тратят :)

\begin{enumerate}
    \item {[10]} Постройте на графике $10$ случайных траекторий изменений богатства Атоса, по горизонтали — 
    номер посещени казино, по вертикали — богатство. 
    На том же графике постройте усреднённую по всем экспериментам тракторию изменения богатства.
    \item {[10]} Постройте на графике $10$ случайных траекторий изменений богатства Портоса, по горизонтали — 
    номер посещени казино, по вертикали — богатство. 
    На том же графике постройте усреднённую по всем экспериментам тракторию изменения богатства.
    \item {[10]} Постройте на графике $10$ случайных траекторий изменений богатства Арамиса, по горизонтали — 
    номер посещени казино, по вертикали — богатство. 
    На том же графике постройте усреднённую по всем экспериментам тракторию изменения богатства.
\end{enumerate}


\item Парадокс Берксона. 

Предположим, что результаты ЕГЭ школьников по русскому и математике независимы и хорошо аппроксимируются 
нормальным распределением с ожиданием 60 и стандартным отклонением 10. 

УШЭ (Урюпинская Школа Экономики) ныне престижна и забирает себе всех школьников набравших более $n$ баллов в сумме по русскому и математике. 

Для каждого $n$ от $80$ до $160$ с шагом $5$ случайно создайте $10^4$ школьников и поделите их на прошедших и не прошедших в УШЭ.

\begin{enumerate}
    \item {[10]} Постройте график числа прошедших в УШЭ в зависимости от $n$. 
    \item {[10]} Постройте график выборочной корреляции между результатами по русскому и математике в зависимости от $n$ 
    среди прошедших в УШЭ.
    \item {[10]} Постройте график выборочной корреляции между результатами по русскому и математике в зависимости от $n$ 
    среди не прошедших в УШЭ.
\end{enumerate}


\end{enumerate}

\newpage
Тем, кто хочет получить бонусные балы и смыть ими тяжкие грехи прошлого\ldots :)

\begin{enumerate}[resume]

\item Парадокс Штайна. 


Ниф-Ниф, Наф-Наф и Нуф-Нуф качают пресс на карантине, чтобы приготовиться к встрече Волка :)
Количества подъёмов туловища в $i$-й день у поросят обозначим $X_i$, $Y_i$ и $Z_i$. 
Эти величины независимы и хорошо аппроксимируются нормальным распределением 
$X_i \sim \cN(60, 100)$, $Y_i \sim \cN(70, 100)$, $Z_i \sim \cN(80, 100)$.

Карантин длится 100 дней. Волк не знает математических ожиданий (60, 70, 80), но знает дисперсии. 
Волку удаётся подсмотреть, сколько раз поросята поднимают свои туловища. 

Проведите $10^4$ симуляций карантина :)

\begin{enumerate}
    \item {[10]} Для каждой симуляции помогите Волку посчитать оценки $\hat\mu_x$, $\hat\mu_y$, $\hat\mu_z$ методом максимального 
    правдоподобия. Постройте гистограмму каждой из оценок и обозначьте на них истинные значения параметров.
    \item {[3]} Постройте гистограмму суммарной квадратичной ошибки, $S = (\hat\mu_x - \mu_x)^2 + (\hat\mu_y - \mu_y)^2 +(\hat\mu_z - \mu_z)^2$.
    Оцените математическое ожидание суммарной квадратичной ошибки. 
    \item {[5]} Отложите оценки Волка $\hat \mu_x$ и $\hat \mu_y$ на диаграмме рассеяния. Найдите их выборочную корреляцию.
\end{enumerate}

Обозначим вектор трёх оценок Волка одной буквой $\hat \mu$. Хитрый Лис тоже охотится на Трёх Поросят. 
Он берёт вектор оценок Волка, домножает его на хитрый множитель, и получает вектор оценок Хитрого Лиса:

\[
\tilde \mu = \left( 1 - \frac{1}{||\hat \mu||^2}\right) \hat \mu    
\]

\begin{enumerate}[resume]
    \item {[10]} Выполните предыдущие три пункта для оценок Хитрого Лиса.
    \item {[2]} Кто точнее оценивает накачанность Ниф-Нифа? Кто точнее оценивает вектор накачанности Трёх Поросят?
\end{enumerate}


\end{enumerate}


\section*{Почитать больше:}

\begin{enumerate}
    \item \url{https://towardsdatascience.com/the-inspection-paradox-is-everywhere-2ef1c2e9d709} 
    \item \url{https://en.wikipedia.org/wiki/Parrondo%27s_paradox}
    \item \url{https://en.wikipedia.org/wiki/Berkson%27s_paradox}
    \item \url{http://www.statslab.cam.ac.uk/~rjs57/SteinParadox.pdf}
\end{enumerate}



\end{document}